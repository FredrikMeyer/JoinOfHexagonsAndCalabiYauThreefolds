\chapter{Triangulations of spheres with 8 vertices}
\label{sec:spheres8}

In the article \cite{grunbaum_enumeration}, Grünbaum--Sreedharan enumerates all simplicial $4$-polytopes with $7$ and $8$ vertices. There are $5$ combinatorial types of triangulations of the $4$-sphere with $7$ vertices, and there are $37$ combinatorial types of triangulations with $8$ vertices.

In her thesis \cite{fausk_thesis}, Ingrid Fausk considered the polytopes with $7$ vertices, and their associated Stanley--Reisner schemes. She showed that four out of the five possible Stanley--Reisner schemes of triangulations of $4$-spheres with seven vertices admit a smoothing. These smoothings correspond to \CY varieties with Hodge numbers $(1,73)$, $(1,73)$, $(1,61)$ and $(1,50)$, respectively. The last one is Rødland's construction.

In this Appendix, we perform deformation theoretic calculations on the $37$ triangulations with $8$ vertices. Unfortunately, most of them appear to be non-smoothable, at least with naïve techniques.

Unfortunately, there seems to be a mistake in Grünbaum--Sreedharan's list. Two of the spheres listed have $H^3(K;k)=0$, which should not occur if they were spheres.

In \cite{kapustka_delpezzo}, Kapustka compiles a list of smooth Calabi--Yau varieties with $\Pic X = \Z$. Several of the smoothings we find below occur in that list. There is also the paper \cite{MR3591944}, where Coughlan--Gołębiowski--Kapustka--Kapustka make a list of arithmetically Gorenstein Calabi--Yau threefolds in $\P^7$, which they conjecture is the complete list of such threefolds. One can ask if all of these are smoothings of one of the Stanley--Reisner schemes from the below list.

\subsection{Technique}

We manually entered the maximal facets from each triangulation $P_i^8$ (in Grünbaum's notation) into \MM. Then we used Nathan Ilten's package \cite{ilten_versaldeformations} to compute their first order deformations and the obstruction spaces, $T^1$ and $T^2$, respectively.

Those with $T^2=0$ are perhaps the most interesting, as they correspond to smooth points on the Hilbert scheme. Having $T^2=0$ means that all first-order deformations lift to a second-order deformation. In many cases this implies that it lifts automatically to an honest family over $\Spec \C[t_1,\ldots,t_N]$ (where $N=\dim_k T^1$).

However, even in non-obstructed cases, we might have power series solutions, meaning that lifting the equations one step at a time will never terminate.

Then we compute the $T^i$ modules for the other triangulations. We also compute their automorphism groups, using SAGE.

\section{Table of information}

Here is the whole table of $T^i$-dimensions together with some other information. Compare with the list in \cite{kapustka_delpezzo}.

%%% no, degree, c_2.H T^1, T^2 Aut(T), funnet glatting?
\begin{center}
\small{
\begin{longtable}{ l >{$}c<{$}  >{$}c<{$} >{$}c<{$}  >{$}c<{$}  >{$}c<{$} c }
Number & \mathrm{degree} & c_2 \cdot H &T^1 & T^2 & \mathrm{Aut}(T) & Comment \\
\hline
\endhead
$P_{1}^8$ & 14 & - & - & - & - & Not a sphere. \\
$P_2^8$   & 14 & 68 & 98 & 9 &  \Z/2 \times \Z/2 \\
$P_3^8$   & 14 & 68 &108 & 24 & D_6 \\
$P_4^8$   & 15 & 66 & 95 & 17 & \Z/2 \times \Z/2 \\
$P_5^8$   & 15 & 64 & 88 & 32 & \Z/2 \times D_4 \\
$P_6^8$   & 15 & 66 & 88 & 9 &  \Z/2 \times \Z/2 \\
$P_7^8$   & 15 & - & - & - & - & Not a sphere.\\
$P_8^8$   & 16 & 64 & 78 & 9 &  1 \\
$P_9^8$   & 16 & 64 & 82 & 17 & \Z/2 \\
$P_{10}^8$& 16 & 64 & 92 & 32 & \Z/4 \\
$P_{11}^8$& 17 & 62 & 74 & 18 & \Z/2 \\
$P_{12}^8$& 17 & 62 & 77 & 25 & \Z/2 \\
$\mathbf{P_{13}^8}$& 15 & 66 & 83 & 0 & S_3 \times D_5 & Smooths to $X_{113} \subset \mathbb G(2,5)$. \\
$P_{14}^8$& 16 & 64 & 80 & 18 & D_4 &  \\
$P_{15}^8$& 16 & 64 & 88 & 32 &\Z/2 \times \Z/4 \\
$\mathbf{P_{16}^8}$& 16 & 64 & 72 & 0 &\Z/2 \times \Z/2  \\
$\mathbf{P_{17}^8}$& 16 & 64 & 72 & 0 &\Z/2 \times \Z/2 \times \Z/2  \\
$P_{18}^8$& 17 & 62 & 72 & 17 & \Z/2 \times \Z/2 \\
$P_{19}^8$& 17 & 62 & 72 & 17 & \Z/2  \\ 
$P_{20}^8$& 17 & 62 & 67 & 9 & \Z/2 \\
$P_{21}^8$& 17 & 62 & 80 & 32 & D_4 \\
$\mathbf{P_{22}^8}$& 17 & 62 & 62 & 0 & \Z/2 \\
$P_{23}^8$& 18 & 60 & 63 & 17 & \Z/2 \\
$P_{24}^8$& 18 & 60 & 18 & 18 & \Z/2  \\
$P_{25}^8$& 18 & 60 & 67 & 25 & 1\\
$\mathbf{P_{26}^8}$& 17 & 62 & 62 & 0 & D_6 \\
$P_{27}^8$& 18 & 60 & 58 & 9 & \Z/2 \\
$P_{28}^8$& 18 & 60 & 58 & 9 & \Z/2 \times \Z/2 \\
$P_{29}^8$& 18 & 60 & 58 & 9 & \Z/2 \times \Z/2 \\
$P_{30}^8$& 19 & 58 & 63 & 33 & \Z/2 \\
$P_{31}^8$& 19 & 58 & 59 & 26 & 1 \\
$P_{32}^8$& 19 & 58 & 55 & 18 & \Z/2\\
$P_{33}^8$& 19 & 58 & 60 & 27 & \Z/2 \times \Z/2 \\
$\mathbf{P_{34}^8}$& 16 & 64 & 72 & 0 & S_4 \times (\Z/2)^4  & Smooths to $X_{2222} \subset \P^7$. \\
$P_{35}^8$& 20 & 56 & 72 & 64 & D_8 \\
$P_{36}^8$& 20 & 56 & 64 & 50 & \Z/4 \\
$P_{37}^8$& 20 & 56 & 61 & 43 & \Z/2 \\
$\mathcal M$& 20 & 56 & 53 & 27 & S_3
\end{longtable}
}
\end{center}

The notations $X_{112}$ and $X_{2222}$ mean a complete intersection of degrees $1,1,2$ (resp. $2,2,2,2$) in $X$. 