\chapter{Introduction}
\label{sec:intro}

The work leading up to this thesis started with a naïve idea concerning smoothings of certain Stanley--Reisner schemes. Stanley--Reisner schemes are highly singular projective schemes, whose components are all projective spaces. They are constructed from a simplicial complex, in such a way that the components correspond to the maximal faces of the simplicial complex. 

If the simplicial complex is homeomorphic to $S^1$, a circle, then a smoothing of the Stanley--Reisner scheme yields an elliptic curve. Similarly, if the simplicial complex is a sphere, a smoothing of the Stanley--Reisner scheme will give a K3 surface. Many properties of the simplicial complex correspond to properties of the Stanley--Reisner scheme and its smoothings. 

The mentioned naïve idea was this: what if the simplicial complex is a triangulated $\C \P^2$? A smoothing of the associated Stanley--Reisner scheme would then give us an (algebraic) hyper-Kähler variety, as we explain in \cref{sec:cp2triangs}. This would be interesting, since there are very few known families of hyper-Kähler varieties.

Unfortunately, given a triangulation of $\C \P^2$ with few vertices, a smoothing of the Stanley--Reisner scheme turned out to be too difficult to find. Even the existence of smoothings are in most cases unclear. However, one particular triangulation of $\C \P^2$ led us to study the problems in Chapter 3 and 4. This triangulation, found by Gaifullin \cite{cp2_15_chess}, is the union of three $4$-balls, all of which are suspensions over joins of hexagons. Leaving the idea of studying triangulations of $\C \P^2$, we began studying 
triangulations of the $3$-sphere.

The join of two hexagons is a triangulated $3$-sphere. A smoothing of the associated Stanley--Reisner scheme $X_0$ is a Calabi--Yau variety. Finding new Calabi--Yau varieties has become a small industry, which we did not hesitate to join. This decision turned out to be profitable. The scheme $X_0$ deforms to several interesting varieties, and three of them are smooth. One of them, which we have denoted by $X_Y$, is a singular Calabi--Yau variety, whose singularities are all locally-analytically cones over del Pezzo-surfaces. This discovery motivates the third chapter, in which we study this singularity and its two smoothings. We prove that they are topologically different, and calculate their Betti numbers.

We construct three smoothings of $X_0$. To define them, recall the definition of \emph{join} of two algebraic varieties. It is the closure of the union of all lines between them. Let $M$ the join of two copies of $\P^2 \times \P^2$ (embedded in disjoint projective spaces). Let $N$ be the join of two copies of $\P^1 \times \P^1 \times \P^1$, and let $W$ be the join of $\P^2 \times \P^2$ and $\P^1 \times \P^1 \times \P^1$. Define $X_1$ to be $M$ intersected by a codimension~$6$ hyperplane. Let $X_2$ be $N$ intersected by a codimension~$4$ hyperplane, and let $X_3$ be $W$ intersected by a codimension~$5$ hyperplane. 

We show that $X_i$ ($i=1,2,3$) are all smooth Calabi--Yau manifolds, and that they are deformations of $X_0$. They have Euler characteristics $-72$, $-48$, and $-60$, respectively. 

To our knowledge, these three Calabi--Yau's have not been previously described. There are many connections to the physics literature, and to works by other mathematicians. Let us explain some of them.

In \cite{kapustka_delpezzo}, the author compiles a list of smooth \CY varieties with $\Pic X = \Z$. One of the elements of the list is a Calabi--Yau in $\P^{11}$ with the same Hilbert polynomial as our $X_1$, and with the Euler characteristic. This Calabi--Yau was however only conjectured to exist, based on the conjecture that to every differential equation of ``Calabi--Yau type'', there should exist a one parameter family of smooth Calabi--Yau varieties having that equation as its Picard--Fuchs differential equation. A list of such equations has been computed by van Enckevort and van Straten in \cite{monodromy_straten}.

All of these equations have been made searchable in the online database \cite{cy_database}. Entering the invariants $H^3=36$, $H \cdot c_2 = 72$ and $\dim |H|=12$, yield exactly three matches, corresponding to Calabi--Yau varities with Euler characteristics $-72$, $-60$ and $-48$, respectively. These numbers are exactly the Euler characteristics of our $X_i$ ($i=1,2,3$).

Furthermore, their differential operators are Hadamard products, $c \ast c$, $a \ast a$ and $a \ast c$, which according to van Straten (personal communication) is ``mirror dual'' to join.

This seems like a perfect match, confirming the existence predicted by the conjecture. The only problem is that our varieties seem to have $h^{11} > 1$.

Several questions arise: can our $X_i$ still correspond to these differential equations, without having $h^{11}=1$? If not, what is their connection to the conjecture?

There also seem to be connections with discoveries made by physicists. For example, Braun--Candelas--Davis describe in \cite{braun_smallhodgenumbers} a Calabi--Yau with small Hodge numbers, whose mirror dual lies in the same deformation family as our $X_i$'s.

We did not have the time to ponder these questions, but would very much like to see them answered in the future.


\hfill \break

Finally, there is the phenomenon of \emph{mirror symmetry}, which is a sort of duality between different Calabi--Yau manifolds. Producing mirror candidates of Calabi--Yau manifolds is a hard problem, and there are many ways to do this. One heuristic which often works is this: suppose you have a family $\pi: \mathscr X \to S$ of Calabi--Yau manifolds, and that some central fiber has a large automorphism group. One can consider the (often singular) sub-family invariant under this group. It is then often the case that a resolution of singularities of an invariant fiber is a mirror to the general fiber of $\pi$. This technique is called \emph{orbifolding}. We give a brief introduction to mirror symmetry and orbifolding in the first chapter.

By using the technique of orbifolding, we produce mirror candidates for $X_1$ and $X_2$.

\hfill \break

The organization of the thesis is as follows:

\begin{itemize}
\item In the first chapter, we gather background material which is relevant for the next chapters. We have erred on the side of \emph{too much} background information rather than too little, serving as a motivation for both myself and potential young readers. We end with a give a brief sketch of some of the ideas from mirror symmetry.

\item In the second chapter we motivate the original naïve idea about smoothing triangulations of $\C \P^2$ to find new hyper-Kähler varieties.

We describe four already known triangulations of $\C \P^2$, with the number of vertices ranging from $9$ to $15$, and describe the obstacles encountered in trying to smooth them. We also compute their associated Stanley--Reisner schemes, and the dimensions of their cotangent modules. Their obstruction spaces are in all cases large.

\item The third chapter is devoted to a special toric singularity, namely the affine cone $C(\dP6)$ over the del Pezzo surface $\dP6$. This singularity has two topologically different smoothings, and we compute their singular homology groups using techniques from toric geometry.

We start the chapter by discussing $\dP6$ in some generality. We discuss its Picard group and two natural embeddings in $\P^1 \times \P^1 \times \P^1$ and $\P^2 \times \P^2$, respectively.

It is well known that $C(\dP6)$ has two smoothings components. We identify them as hyperplane complements of $\P^1 \times \P^1 \times \P^1$ and of $\P(\mathcal T_{\P^2})$, and use this fact to compute their singular homology groups. In the final computation, we use theorems from algebraic topology, such as Poincaré duality and Lefschetz duality. 

\item The final chapter is devoted to the construction of new Calabi--Yau varieties and their mirror candidates.

We start the chapter by discussing the Stanley--Reisner scheme $X_0$, which comes from the simplicial complex that is the join of two hexagons. We compute its Hilbert polynomial, and explain how it deforms to a special singular Calabi--Yau variety $X_Y$.

Then we explain the construction of three topologically different smoothings $X_i$ ($i=1,2,3$) of $X_Y$ (and hence of $X_0$). They are topologically different, which we prove using a \MM computation: it shows that their topological Euler characteristics are different. The construction is very similar to that of Rødland \cite{rodland_pfaffian}.

Then we explain the existence of special singular subfamilies of $X_1$ and $X_2$ which are invariant under a finite subgroup of the big torus. Using {orbifolding} and a formula by Roan \cite{roan_euler}, we propose conjectural mirror candidates for $X_1$ and $X_2$.

We end with many open questions, which we hope to see answered in the future.
\end{itemize}

In the last appendix we include some computations on triangulations of spheres with $8$ vertices. Grünbaum and Sreedharan have computed all such triangulations \cite{grunbaum_enumeration}, and we used his list to compute deformation theoretic invariants for each of the associated Stanley--Reisner schemes of the spheres with $8$ vertices. unfortunately, there seem to be a few typographical errors in his article, as some of the complexes in his list turn out not to be spheres.

The source code of the thesis and all computer computations are available on GitHub at \url{https://github.com/FredrikMeyer/Matematikknotater/tree/master/thesis} and \url{https://github.com/FredrikMeyer/m2files}, respectively.

%%%%%%
%%%%%%
\section{Notation}

If $V$ is a vector space, we denote by $\P(V)$ its projectivisation. We write $k$ for a field, which is almost always assumed to be $\C$. If $X$ is a projective variety, we write $S(X)$ for its homogeneous coordinate ring (if the embedding is implicit). If $X$ is a scheme over $k$, we write $X/k$. We will write $h^i(X,\mathscr F)$ for $\dim_k H^i(X,\mathscr F)$. All schemes are noetherian. We will often write $\stackrel \Delta = $ for definitions (instead of  ``$:=$'', common in computer science literature). Unless otherwise stated, we use the definitions from \cite{hartshorne}.