\begin{frame}
\frametitle{Hodge numbers}

\begin{itemize}
	\item The quintic $X=V(f) \subset \P^4$ is the canonical example of a Calabi--Yau. It has Hodge numbers $h^{11}=1$ and $h^{12}=101$.
\end{itemize}

\begin{remark}[Heuristic]
The number $h^{12}$ is the dimension of the ``space of parameters'' of $X$. The following heuristic will give us the correct Hodge number:
\begin{itemize}
	\item The space of degree $5$ polynomials $H^0(\P^4,\OO_{\P^4}(5))$ in $\P^4$ is $\binom{4+5}{5}=\binom{9}{4}=126$-dimensional. Hence $\P(H^0(\P^4,\OO_{\P^4}(5)) = \P^{125}$.
	\item This is not unique, but we can act by $\PGL(5)$ to identify isomorphic quintics. We have $\dim \PGL(5)=25-1=24$.
	\item In total: $125-24=101$, which is $h^{12}(X)$.
\end{itemize}
\end{remark}

\end{frame}