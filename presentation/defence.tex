\documentclass[UKenglish]{beamer}


\usetheme{MathDept}
\usepackage{presentation}
\usepackage{style}


\day = 8
\month = 11
\year = 2017

\title{Join of hexagons and {Calabi--Yau} threefolds}
\subtitle{Public defence}
\author{Fredrik Meyer}


\begin{document}

\section[Section]{My section}
\begin{frame}
\frametitle{Outline of the thesis}

\begin{itemize}
	\item Unsuccessful attempt to find new hyper-Kähler varieties. \pause
	\item The topology of $C(\dP6)$. \pause
	\item New Calabi--Yau varieties and potential mirror partners.
\end{itemize}

\end{frame}

\begin{frame}
\frametitle{Outline of the thesis}

    \begin{textblock}{0.3}(0.25, 0.22)
        Dette er en boks hvis bredde er \alert{0.3} av skjermbredden, og der hjørnet øverst til venstre er \alert{0.25} av skjermbredden fra venstre kant og \alert{0.22} av skjermhøyden fra toppen.
    \end{textblock}

\end{frame}

\begin{frame}
\frametitle{Calabi--Yau manifolds}

\begin{definition}[Calabi--Yau variety]
A Calabi--Yau variety is a smooth projective scheme $X/\C$ of dimension $3$ satisfying:
\begin{itemize}
	\item $H^0(X,\OO_X)=H^3(X,\OO_X)=k$ and $h^1(X,\OO_X)=h^2(\OO_X)=0$.
	\item The canonical sheaf is trivial: $\omega_X \simeq \OO_X$. 
\end{itemize}
\end{definition}

\end{frame}


\end{document}