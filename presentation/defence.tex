\documentclass[UKenglish]{beamer}


\usetheme{MathDept}
\usepackage{presentation}
\usepackage{style}


\day = 8
\month = 11
\year = 2017

\title{Join of hexagons and {Calabi--Yau} threefolds}
\subtitle{Public defence}
\author{Fredrik Meyer}


\begin{document}

\section[Section]{My section}
\begin{frame}
\frametitle{Outline of the thesis}

\begin{itemize}
	\item Unsuccessful attempt to find new hyper-Kähler varieties. \pause
	\item The topology of $C(\dP6)$. \pause
	\item New Calabi--Yau varieties and potential mirror partners.
\end{itemize}

\end{frame}

\begin{frame}
\frametitle{Outline of the thesis}

    \begin{textblock}{0.3}(0.25, 0.22)
        Dette er en boks hvis bredde er \alert{0.3} av skjermbredden, og der hjørnet øverst til venstre er \alert{0.25} av skjermbredden fra venstre kant og \alert{0.22} av skjermhøyden fra toppen.
    \end{textblock}

\end{frame}

\begin{frame}
\frametitle{Calabi--Yau manifolds}

\begin{definition}[Calabi--Yau variety]
A Calabi--Yau variety is a smooth projective scheme $X/\C$ of dimension $3$ satisfying:
\begin{itemize}
	\item $H^0(X,\OO_X)=H^3(X,\OO_X)=k$ and $h^1(X,\OO_X)=h^2(\OO_X)=0$.
	\item The canonical sheaf is trivial: $\omega_X \simeq \OO_X$. 
\end{itemize}
\end{definition}

\begin{textblock}{0.5}(0.0, 0.6)
\begin{itemize}
	\item Easiest invariant are the Euler characteristic and the Hodge numbers.

	\item We always have $\chi = 2(h^{11}-h^{12})$. 
\end{itemize}
\end{textblock}

    \begin{textblock}{0.4}(0.60, 0.58)
    \[
    \only<1>{
   \arraycolsep=1pt\def\arraystretch{0.5}
\begin{array}[c]{ccccccc}
&&&               h^{00} \\  
 &         & h^{01} & & h^{10}  \\
&	h^{02} && h^{11}  && h^{20}  \\
   h^{03} && h^{12} && h^{21} && h^{30} \\
&	h^{13} && h^{22}  && h^{31}  \\
 &         & h^{23} & & h^{32}  \\
                &&& h^{33} 
\end{array}
}
\only<2>{
\arraycolsep=1.5pt\def\arraystretch{1}
\begin{array}[c]{ccccccc}
&&& 1 \\  
 &         & 0 & & 0  \\
&	0 && h^{11}  && 0  \\
1 && h^{12} && h^{12} && 1 \\
&	0 && h^{11}  && 0  \\
 &         & 0 & & 0  \\
 &&& 1 
\end{array}
}
\]

    \end{textblock}

\end{frame}

\begin{frame}
\frametitle{Hodge numbers}

\begin{itemize}
	\item The quintic $X=V(f) \subset \P^4$ is the canonical example of a Calabi--Yau. It has Hodge numbers $h^{11}=1$ and $h^{12}=101$.
\end{itemize}

\begin{remark}[Heuristic]
The number $h^{12}$ is the dimension of the ``space of parameters'' of $X$. The following heuristic will give us the correct Hodge number:
\begin{itemize}
	\item The space of degree $5$ polynomials $H^0(\P^4,\OO_{\P^4}(5)$ in $\P^4$ is $\binom{4+5}{5}=\binom{9}{4}=126$-dimensional. Hence $\P(H^0(\P^4,\OO_{\P^4}(5)) = \P^{125}$.
	\item This is not unique, but we can act by $\PGL(5)$ to identify isomorphic quintics. We have $\dim \PGL(5)=25-1=24$.
	\item In total: $125-24=101$, which is $h^{12}(X)$.
\end{itemize}
\end{remark}

\end{frame}


\begin{frame}
\frametitle{Mirror symmetry}


\end{frame}

\end{document}