\begin{frame}
\frametitle{Mirror symmetry}

\begin{itemize}[<+->]
	\item
    \CY threefolds seem to ``always'' have ``mirror partners''.

	\item
    Mirror partner $X^\circ$ to $X$ has ``mirrored Hodge diamond''.

	\item
    Hence $\chi(X^\circ) = - \chi(X)$.
\end{itemize}

\unskip
\begin{columns}[onlytextwidth]
    \only<1->
    {
        \begin{column}{0.48\textwidth}
            \[
                \begin{array}[c]{ccccccc}
                    &&& X                    \\
                    \hline                   \\[-1.8ex]
                    &&& 1                    \\
                    &&  0 && 0               \\
                    &   0 && 1   && 0        \\
                        1 && 101 && 101 && 1 \\
                    &   0 && 1   && 0        \\
                    &&  0 && 0               \\
                    &&& 1
                \end{array}
            \]
        \end{column}
    }

    \only<2->
    {
        \begin{column}{0.48\textwidth}
            \[
                \begin{array}[c]{ccccccc}
                    &&& \phantom{^\circ}X^\circ \\
                    \hline                      \\[-1.8ex]
                    &&& 1                       \\  
                    &&  0 && 0                  \\
                    &   0 && 101 && 0           \\
                        1 && 1   && 1 && 1      \\
                    &   0 && 101 && 0           \\
                    &&  0 && 0                  \\
                    &&& 1
                \end{array}
            \]
        \end{column}
    }
\end{columns}

\end{frame}

\begin{frame}
\frametitle{The mirror construction Ansatz}

Sometimes the following method produces a mirror manifold of a \CY $X$:

\begin{enumerate}
    \alert<2-4>
    {
        \item
        \alt<1>
        {Suppose $X$ has a natural degeneration $X_0$ with a finite automorphism group $G$.}
        {The general quintic degenerates to the singular scheme $V(x_0x_1x_2x_3x_4)$.}
    }

    \alert<3-4>
    {
        \item
        \alt<1-2>
        {Find a family $\pi \colon \mathscr{X} \to S$ on which $G$ act, and such that the general fiber $X_t$ has only isolated singularities.}
        {The family defined by $f_t=x_0x_1x_2x_3x_4 + t \sum_{i = 1}^5 x_i^5$ is $S_5$-invariant.}
    }

    \alert<4-4>
    {
        \item
        \alt<1-3>
        {
            There might be a finite subgroup $H$ of the big torus acting.
            \par
            A mirror candidate is then a crepant resolution of $X_t/H$.
        }
        {
            There is an action of $H \defeq (\Z/5)^5/(\Z/5)$ on $X_t$.
            \par
            Crepant resolutions of $X_t/H$ exists, and are mirrors.
        }
    }
\end{enumerate}


\only<5->
{
    We can use \alert{Roan's formula} to compute the Euler characteristic:

    \begin{theorem}[Roan's formula]
    \[
        \chi\Big(\widetilde{X_t/H}\Big)
        =
        \frac 1{|H|} \sum_{g,h \in H} \chi \Big(X_t^g \cap X_t^h\Big)
    \]
    \end{theorem}
}

\end{frame}