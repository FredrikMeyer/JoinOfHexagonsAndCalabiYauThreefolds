\begin{frame}
\frametitle{Calabi--Yau varieties}

\begin{definition}[Calabi--Yau variety]
A Calabi--Yau variety is an irreducible smooth projective scheme $X/\C$ of dimension $3$ satisfying:
\begin{itemize}
	\item $H^0(X,\OO_X)=H^3(X,\OO_X)=\C$ and $H^1(X,\OO_X)=H^2(X,\OO_X)=0$.
	\item The canonical sheaf is trivial: $\omega_X \simeq \OO_X$. 
\end{itemize}
\end{definition}

\begin{textblock}{0.5}(0.0, 0.6)
\begin{itemize}
	\item Easiest invariants are the Euler characteristic and the Hodge numbers. $h^{ij}=h^j(X,\Omega_{X/\C}^i)$.

	\item We always have $\chi = 2(h^{11}-h^{12})$. 
\end{itemize}
\end{textblock}

    \begin{textblock}{0.4}(0.60, 0.58)
    \[
    \only<1>{
   \arraycolsep=1pt\def\arraystretch{0.5}
\begin{array}[c]{ccccccc}
&&&               h^{00} \\  
 &         & h^{01} & & h^{10}  \\
&	h^{02} && h^{11}  && h^{20}  \\
   h^{03} && h^{12} && h^{21} && h^{30} \\
&	h^{13} && h^{22}  && h^{31}  \\
 &         & h^{23} & & h^{32}  \\
                &&& h^{33} 
\end{array}
}
\only<2>{
\arraycolsep=1.5pt\def\arraystretch{1}
\begin{array}[c]{ccccccc}
&&& 1 \\  
 &         & 0 & & 0  \\
&	0 && h^{11}  && 0  \\
1 && h^{12} && h^{12} && 1 \\
&	0 && h^{11}  && 0  \\
 &         & 0 & & 0  \\
 &&& 1 
\end{array}
}
\]

\end{textblock}
\end{frame}


\begin{frame}
\frametitle{Hodge numbers}

\begin{itemize}
  \item The quintic $X=V(f) \subset \P^4$ is the canonical example of a Calabi--Yau. It has Hodge numbers $h^{11}=1$ and $h^{12}=101$.
\end{itemize}

\begin{remark}[Heuristic]
The number $h^{12}$ is the dimension of the ``space of parameters'' of $X$. The following heuristic will give us the correct Hodge number:
\begin{itemize}
  \item The space of degree $5$ polynomials $H^0(\P^4,\OO_{\P^4}(5))$ in $\P^4$ is $\binom{4+5}{5}=\binom{9}{4}=126$-dimensional. Hence $\P \left(H^0(\P^4,\OO_{\P^4}(5))\right) = \P^{125}$.
  \item This is not unique, but we can act by $\PGL(5)$ to identify isomorphic quintics. We have $\dim_\C \PGL(5)=25-1=24$.
  \item In total: $125-24=101$, which is $h^{12}(X)$.
\end{itemize}
\end{remark}

\end{frame}