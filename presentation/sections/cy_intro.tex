\begin{frame}
\frametitle{\CY varieties}

\begin{definition}%[\CY variety]
    A \alert{\CY variety} is an irreducible, smooth, projective scheme $X/\C$ of dimension~$3$ satisfying:
    \begin{itemize}
      \item
      $H^0(X, \OO_X) = H^3(X, \OO_X) = \C$ and $H^1(X, \OO_X) = H^2(X, \OO_X) = 0$.

      \item
      The canonical sheaf is trivial: $\omega_X \simeq \OO_X$. 
    \end{itemize}
\end{definition}

\unskip
\begin{columns}[onlytextwidth]
    \begin{column}{0.6\textwidth}
        \begin{itemize}
          \item
          Easiest invariants are the {Euler characteristic} and the {Hodge numbers}, $h^{ij} = h^j\left(X, \Omega_{X/\C}^i\right)\mkern-2mu$.

            \item
            We always have $\chi = 2(h^{11} - h^{12})$. 
        \end{itemize}
    \end{column}
\end{columns}

\only<1>
{
    \begin{textblock}{0.4}(0.58, 0.58)
        \[
            \arraycolsep = 1pt
            \def\arraystretch{0.5}
            \begin{array}[c]{ccccccc}
                &&& h^{00}                               \\  
                &&  h^{01} && h^{10}                     \\
                &   h^{02} && h^{11} && h^{20}           \\
                    h^{03} && h^{12} && h^{21} && h^{30} \\
                &   h^{13} && h^{22} && h^{31}           \\
                &&  h^{23} && h^{32}                     \\
                &&& h^{33} 
            \end{array}
        \]  
    \end{textblock}
}

\only<2>
{
    \begin{textblock}{0.4}(0.58, 0.58)
        \[
            \arraycolsep = 1pt%.5pt
            \def\arraystretch{0.5}%7}
            \begin{array}[c]{ccccccc}
                \phantom{h^03}  & \phantom{h^{02}} & \phantom{20} & 1 \\
                &&  0 && 0      & \phantom{h^30}                      \\
                &   0 && h^{11} && 0                                  \\
                    1 && h^{12} && h^{12} && 1                        \\
                &   0 && h^{11} && 0                                  \\
                &&  0 && 0 \vphantom{h^{32}}                          \\
                &&& 1 &&   \vphantom{h^{33}}
            \end{array}
        \]
    \end{textblock}
}
\end{frame}

\begin{frame}
\frametitle{Hodge numbers}

The quintic $X = V(f) \subset \P^4$ is the canonical example of a \CY. It has Hodge numbers $h^{11}=1$ and $h^{12}=101$.

\begin{remark}[Heuristic]
    The number $h^{12}$ is the dimension of the ``space of parameters'' of $X$. The following heuristic will give us the correct Hodge number:
    \begin{itemize}
      \item
      The space of degree $5$ polynomials $H^0\big(\P^4, \OO_{\P^4}(5)\big)$ in $\P^4$ is $\binom{4 + 5}{5} = \binom{9}{4} = 126$-dimensional. Hence $\P\big(H^0\big(\P^4, \OO_{\P^4}(5)\big)\big) = \P^{125}\mkern-4mu$.
      \pause

      \item
      This is not unique, but we can act by $\PGL(5)$ to identify isomorphic quintics. We have $\dim \PGL(5)=25-1=24$.
      \pause

      \item
      In total: $125 - 24 = 101$, which is $h^{12}(X)$.
      \pause
    \end{itemize}
\end{remark}

\end{frame}