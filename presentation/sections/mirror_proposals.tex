\begin{frame}
\frametitle{Mirror candidate for $X_1$}

Using the mirror Ansatz, we propose mirror candidates for $X_1$ and $X_2$.

\begin{itemize}[<+->]
	\item
	There is an $H \defeq \Z/3$-action on $E$ defined by $e_i \mapsto \omega^i e_i$.

	\item
	Another $H$-action $e_i \mapsto e_{i + 1}$.

	\item
	Extend to actions on $\P\big((E \otimes E) \oplus (E \otimes E)\big) = \P^{17}\mkern-4mu$.

	\item
	Choose invariant $\P^{11}$: Defined by
	\[
	    f_{ij}^\alpha = e_{ij}^\alpha + t_{-i-j}^\alpha e_{-i-j,-i-j}^{\alpha+1}
	\]
	for $i,j \in \Z/3 \times \Z/3$ ($i \neq j$) and $\alpha = 0,1$.

	\item
	The resulting $X_{H_t} \defeq \P^{11} \cap M$ is singular with $48$ isolated double point singularities.
\end{itemize}

\end{frame}

\begin{frame}
\frametitle{Mirror candidate for $X_1$}

\begin{itemize}
	\item We divide out by the $H$-action and resolve: $X_1^\circ \defeq \widetilde{X_{H_t}/H}$.
	\pause
	\item Roan's formula gives:
	\[
	    \chi(X_1^\circ) = \frac{1}{3} \left(24 + 8 \cdot 24\right) = 72.
	\]
\end{itemize}

Based on this calculation and the mirror heuristic, we conjecture:

\begin{conjecture}
    $X_1^\circ$ is a mirror of $X_1$.
\end{conjecture}

\pause
\begin{remark}
    A very similar construction gives a mirror candidate for $X_2$.
\end{remark}

\end{frame}
