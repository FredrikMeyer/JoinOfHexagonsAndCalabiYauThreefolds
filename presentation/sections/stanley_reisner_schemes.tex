\begin{frame}
\frametitle{Stanley--Reisner schemes}


\begin{itemize}
    \item
    Given a simplicial complex $\F$, we get a Stanley--Reisner scheme~$\P(\F)$.
    \pause

    \item
    Is a union of projective spaces $\P^{\dim f}$, indexed by faces $f$ of $\F$.
    \pause
\end{itemize}


\begin{example}

From a simplicial complex to a union of $\P^1$'s.

\begin{center}
\unskip
\vskip1ex
\tikzstyle{ball} = [circle,shading=ball, ball color=uiored,
    minimum size=0.01cm]
\begin{tikzpicture}
\coordinate (A) at (2.5,-1.2);
\coordinate (B) at (4.1, -1.2);
\coordinate (C) at (5, 0);
\coordinate (D) at (4.1, 1.2);
\coordinate (E) at (2.5, 1.2);
\coordinate (F) at (1.6, 0);

\draw (A) -- (B) -- (C) -- (D) -- (E) -- (F) -- cycle;

\node[style=ball, scale=0.5] at (A) {};
\node[style=ball, scale=0.5] at (B) {};
\node[style=ball, scale=0.5] at (C) {};
\node[style=ball, scale=0.5] at (D) {};
\node[style=ball, scale=0.5] at (E) {};
\node[style=ball, scale=0.5] at (F) {};

\draw (6.8,-1) -- (9.2,-1); % E_1
\draw (8.4,-1.2) -- (9.7,0.3); % L_12
\draw (7.6,-1.2) -- (6.3,0.3); % L_13
\draw (8.4,1.2) -- (9.7,-0.3);
\draw (7.6,1.2) -- (6.3,-0.3);
\draw (6.8,1) -- (9.2, 1); % L_23
\end{tikzpicture}
\end{center}
\unskip

The ideal is generated by $x_ix_{i+2}=x_ix_{i+3}=0$ ($i=0,\ldots,5$).

\end{example}
\end{frame}


\begin{frame}
\frametitle{Stanley--Reisner schemes}

\begin{itemize}
	\item
	\alert{Join} $X \ast Y$ of two projective subschemes $X$ and $Y$: The (closure of the) union of all lines between $X$ and $Y$.
	\pause

	\item
	\alert{Join} of two Stanley--Reisner schemes $\P(\F)$ and $\P(\mathcal G)$ is $\P(\F \ast \mathcal G)$, where the faces of $\F  \ast \mathcal G$ are $f \sqcup g$ for $f \in \F$, $g \in \mathcal G$.
\end{itemize}

\pause

Smoothings of Stanley--Reisner schemes:

\begin{itemize}
	\item Given a basis for $T^1(S_{\P(\K)}/k,S_{\P(\K)})_0$, we can try to find a smoothing of $X_0 = \P(\K)$.
	\pause
	\item A smoothing $X$ of $X_0$ will have many of the same properties:
		\begin{itemize}
			\item The same Hilbert polynomial.
			\item By semicontinuity, if $X_0$ is a sphere, $X$ will be \CY. %% which we define now
		\end{itemize}
\end{itemize}


\end{frame}
