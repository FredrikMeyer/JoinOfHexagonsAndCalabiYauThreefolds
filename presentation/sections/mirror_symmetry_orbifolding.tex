\begin{frame}
\frametitle{The orbifold heuristic}

Sometimes the following method produces a mirror manifold of a Calabi--Yau $X$.

\begin{enumerate}
	\item \only<1>{Suppose $X$ has a natural degeneration $X_0$ with a finite automorphism group $G$.}
	\only<2->{
	\alert{The general quintic degenerates to the singular scheme $V(x_0x_1x_2x_3x_4)$.}} % Rød farge?

	\item \only<1-2>{Find a family $\pi: \mathscr X \to S$ on which $G$ act, and such that the general fiber $X_t$ have only isolated singularities.}
	\only<3->{
	\alert{
		The family defined by $f_t=x_0x_1x_2x_3x_4 +t \sum_{i=1}^5 x_i^5$ is $S_5$-invariant.
	}
	}

	\item \only<1-3>{There might be a finite subgroup $H$ of the big torus acting. A mirror candidate is then a crepant resolution of $X_t/H$.}
	\only<4->{
	\alert{
		There is an action of $H \stackrel \Delta = (\Z/5)^5/(\Z/5)$ on $X_t$. Crepant resolutions of $X_t/H$ exists, and is a mirror.
	}
	}
\end{enumerate}


\only<5->{
We can use \emph{Roan's formula} to compute the Euler characteristic.

\begin{theorem}[Roan's formula]
$$
\chi(\widetilde{X_t/H}) = \frac 1{|H|} \sum_{g,h \in H} \chi \left(X_t^g \cap X_t^h\right).
$$
\end{theorem}
}
\end{frame}