\begin{frame}
\frametitle{The orbifold heuristic}

Sometimes the following method produces a mirror manifold of a \CY $X$:

\begin{enumerate}
    \alert<2-4>
	{
	    \item
	    \alt<1>
	    {Suppose $X$ has a natural degeneration $X_0$ with a finite automorphism group $G$.}
	    {The general quintic degenerates to the singular scheme $V(x_0x_1x_2x_3x_4)$.}
	}

    \alert<3-4>
    {
	    \item
	    \alt<1-2>
	    {Find a family $\pi \colon \mathscr{X} \to S$ on which $G$ act, and such that the general fiber $X_t$ have only isolated singularities.}
	    {The family defined by $f_t=x_0x_1x_2x_3x_4 + t \sum_{i = 1}^5 x_i^5$ is $S_5$-invariant.}
	}

    \alert<4-4>
    {
	    \item
	    \alt<1-3>
	    {
	        There might be a finite subgroup $H$ of the big torus acting.
	        \par
	        A mirror candidate is then a crepant resolution of $X_t/H$.
	    }
	    {
	        There is an action of $H \stackrel \Delta = (\Z/5)^5/(\Z/5)$ on $X_t$.
	        \par
	        Crepant resolutions of $X_t/H$ exists, and is a mirror.
	    }
	}
\end{enumerate}


\only<5->
{
    We can use \alert{Roan's formula} to compute the Euler characteristic:

    \begin{theorem}[Roan's formula]
    \(
        \displaystyle
        \chi\Big(\widetilde{X_t/H}\Big)
        =
        \frac 1{|H|} \sum_{g,h \in H} \chi \Big(X_t^g \cap X_t^h\Big)
    \)
    \end{theorem}
}

\end{frame}