\begin{frame}
\frametitle{Construction of a new \CY: $X_1$}

\begin{itemize}
	\item Let $E$ be a vector space with basis $e_1,e_2,e_3$. Consider
	$$
	\P^{17} = \P(E \otimes E \oplus E \otimes E).
	$$
	The elements are pairs of $3 \times 3$ matrices, equal up to scalar multiplication.
	\item Consider the set of pairs $M$ of matrices $(A,B)$ with rank $1+1$. 
	\item Intersect $M$ with a generic $\P^{11} \subset \P^{17}$. Let $X_1 \stackrel \Delta = M \cap \P^{11}$.
\end{itemize}

\begin{theorem}
$X_1$ is a smooth \CY with Euler-characteristic $-72$.
\end{theorem}

\end{frame}


\begin{frame}
\frametitle{Construction of a new \CY: $X_2$}

\begin{itemize}
	\item Let $F$ be a vector space with basis $f_1,f_2$. Consider
	\[
        \P^{15} = \P(F \otimes F \otimes F \oplus F \otimes F \otimes),
        \alert{????}
	\]
	The elements are pairs of $2 \times 2 \times 2$-tensors, equal up to scalar multiplication.
	\item Consider the set of pairs $N$ of tensors $(A,B)$ with rank $1+1$.
	\item Intersect $N$ with a generic $\P^{11} \subset \P^{15}$. Let $X_2 \stackrel \Delta = N \cap \P^{11}$.
\end{itemize}

\begin{theorem}
$X_2$ is a smooth \CY with Euler-characteristic $-48$.
\end{theorem}

\end{frame}

\begin{frame}
\frametitle{Construction of a new \CY: $X_3$}

\begin{itemize}
	\item Let $E$ and $F$ be as before. Consider
	$$
\P^{16} = \P(E \otimes E \oplus F \otimes F \otimes).
	$$
	\item Consider the set of pairs $W$ of tensors $(A,B)$ with rank $1+1$.
	\item Intersect $W$ with a generic $\P^{11} \subset \P^{16}$. Let $X_3 \stackrel \Delta =  W \cap \P^{11}$.
\end{itemize}

\begin{theorem}
$X_3$ is a smooth \CY with Euler-characteristic $-60$.
\end{theorem}

\end{frame}

\begin{frame}
\frametitle{Hodge number heuristics}
\begin{conjecture}
$X_1$ have Hodge numbers $h^{11}=3$ and $h^{12}=39$.
\end{conjecture}
\begin{proof}[``Reason'']

\begin{enumerate}[<+->]
	\item We know the Euler characteristic, so it is enough to find $h^{12}$ ($=$ number of parameters).
	\item The Grassmannian of $\P^{11}$'s in $\P^{17}$ is $72$-dimensional.
	\item We can act by automorphism from $\prod_{i=1}^4 \GL(E)$.
	\item The subgroup $\{ t_1t_2=t_3t_4 \} \subset (\C^\ast)^4 \subset \prod_{i=1}^4 \GL(E)$ acts trivially.
	\item Hence $h^{12}=72-(\dim \prod_{i=1}^4 \GL(E) - 3) = 72-33=39$.\qedhere
\end{enumerate}

\end{proof}


\end{frame}