\begin{frame}
\frametitle{Abelian varieties}

\begin{itemize}
	\item Named from Niels Henrik Abel (1802-1829).
	\item Integration of
	 \[
	\int_\gamma \frac{dz}{\sqrt{f(z)}}.
	\]
	$\gamma$ is a path in $\C$. \pause
	\item Enter Riemann surfaces: $C=\{ (w,z) \mid w^2=f(z) \}$.
	\item Let 
	\[
	\omega_i = z^{i-1} \frac{dz}{\sqrt{f(z)}}, \, \, i=1,\ldots,g
	\]
	be a basis for $H^0(C,\omega_C)$.
\end{itemize}

\end{frame}


\begin{frame}
\frametitle{The Albanese map}

Fix a point $p_0 \in C$ and consider
\[
p \mapsto \left(
\int_{p_0}^p \omega_1, \ldots,  \int_{p_0}^p \omega_g
\right)
\]
defined on a small $U \ni p_0$.
\pause

Cannot be extended to all of $C$, but well-defined modulo $H_1(C,\Z)$:

\begin{align*}
\alpha_{p_0}:C &\to H^0(C,\omega_C)^\ast/ H_1(C,\Z) =: J(C) \\
p &\mapsto \int_{p_0}^p \omega
\end{align*}

Isomorphic to $(S^1)^{2g}$.
\end{frame}

\begin{frame}
\frametitle{Abelian varieties in general}

\begin{itemize}
	\item Over $\C$, $X=\C^n/\Lambda$, where $\Lambda$ is lattice of full rank (i.e. $\Lambda \otimes_\Z \R = \C^n$) and such that $\Lambda$ satisfies the \emph{Riemann relations}.
	\item They are all projective varieties.
	\item The \alert{dual abelian variety}:
	\begin{align*}
	\hat X &= (\C^n)^\ast/ \Lambda ^\ast \\
	&=\{ \text{line bundles $\L$ with $c_1(\L)=0$} \} &(= \Pic^0(X)).
	\end{align*}
	\pause
	\item Given $\L \in \Pic(X)$, we get a map $\varphi_\L$
	\[
	X \ni x \stackrel{\varphi_\L}{\mapsto} t_x^\ast \L \otimes L^{-1} \in \Pic^0(X) = \hat X.
	\]
\end{itemize}

\end{frame}

\begin{frame}
\frametitle{Abelian varieties in general}

\begin{itemize}
	\item A \alert{polarization} on $X$ is a morphism $\rho: X \to \hat X$ such that $\rho=\varphi_\mathcal L$ for some \emph{ample} $\mathcal L$.
	\item It is \alert{principal} if it is an isomorphism. \pause
	\item If it is principal, then $H^0(X, \L)=\sigma \C$, and the divisor-of-zeroes of $\sigma$ is then a \alert{$\Theta$-divisor}.
	\pause
	\item Principally polarized abelian varieties = nice \alert{$\ddot\smile$} $\Rightarrow$ well behaved moduli spaces.
\end{itemize}
\end{frame}