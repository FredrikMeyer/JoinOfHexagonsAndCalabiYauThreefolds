\begin{frame}
\frametitle{A few words about Jacobians}

\begin{itemize}
	\item The intersection form on $H_1(C\Z)$ induces a principal polarization, hence we get a \alert{$\Theta$-divisor} $\Theta_C \subset J(C)$.
	\item Geometrically: Consider the Abel--Jacobi morphism
	\begin{align*}
	\varphi_d\colon C^{(d)} &\to J(C) \\
	(x_1,\ldots,x_d) &\mapsto [x_1+\cdots+ x_d-dp_0].
	\end{align*}
	Then $\overline{\varphi_{g-1}\big(C^{g-1}\big)}$ is (up to translation) the $\Theta$-divisor. \pause
	\item The Jacobian $J(C)$ is universal in the following sense: If $C \xrightarrow{\varphi} X$ is any morphism sending $p_0 \in C$ to $0 \in X$ ($X$ abelian), then it factors uniquely through $\alpha_{p_0}$.
\end{itemize}
\end{frame}