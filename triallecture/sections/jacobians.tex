\begin{frame}
\frametitle{A few words about Jacobians}

\begin{itemize}
	\item The intersection form on $H_1(C\Z)$ induces a principal polarization, hence we get a \alert{$\Theta$ divisor} $\Theta_C \subset J(C)$.
	\item Geometrically: consider the Abel-Jacobi morphism
	\begin{align*}
	\varphi_d: C^{(d)} &\to J(C) \\
	(x_1,\ldots,x_d) &\mapsto [x_1+\ldots x_d-dc_0].
	\end{align*}
	Then $\overline{\varphi_{g-1}(C^{g-1})}$ is (up to translation) the $\Theta$ divisor. \pause
	\item The Jacobian $J(C)$ are universal in the following sense: if $C \xrightarrow{\varphi} X$ is any morphism sending $c_0 \in C$ to $0 \in X$ ($X$ abelian), then it factors uniquely through $\alpha_c$.
\end{itemize}
\end{frame}