\documentclass[11pt, english]{article} 
\usepackage[utf8]{inputenc}
\usepackage[T1]{fontenc}
\usepackage[english]{babel}   % S P R A A K


% \usepackage{graphicx}    % postscript graphics
\usepackage{amssymb, amsmath, amsthm, amssymb} % symboler, osv
\usepackage[poly,matrix]{xy}   % for comm.diagram
\usepackage{url}
\usepackage{thmtools}
\usepackage{style} % stilfilen
\usepackage{enumerate}  % lister $
\usepackage{float}
\usepackage{tikz}

\title{Definitions in preparation of trial lecture}
\author{Fredrik Meyer}
\date{}
\begin{document}  
\maketitle

We assume $k = \overline{k}$ or $k=\C$.

\section{Problems and theorems}

\begin{itemize}
	\item \textbf{Schottky problem}

	$(A,\Theta)$ principally polarized abelian variety. Find geometric conditions on $(A,\Theta)$ for it to be isomorphic the Jacobian of non-singular genus $g$ curve $C$.
	\item \textbf{Riemann's singularity theorem}

	For every $L \in \Pic^{(g-1)}(C)$, we have $\mathrm{mult}_L W_{g-1} = h^0(L)$.
	\item \textbf{Torelli theorem}

	If $(J(C), \Theta)$ and $(J(C'),\Theta')$ are isomorphic as principally poralized abelian varieties, then $C \approx C'$.
\end{itemize}

\section{Definitions}

\subsubsection{Dual abelian variety} % (fold)
\label{ssub:dual_abelian_variety}

Analytically, if $X$ is $\C^n/\Lambda$, then $\hat X$ is ${\C^n}^\ast/\Lambda^\ast$,  where $\Lambda^\ast$ is the space of $\Z$-valued forms on $\Lambda$.

% subsubsection dual_abelian_variety (end)

\subsection{Isogeny}

An \emph{isogeny} of algebraic groups is a surjective morphism with finite kernel.

\subsection{Jacobian variety of curve $C$}

Algebraic definition: the group $\Pic^0 (C)$ of line bundles of degree zero.

Analytic definition: $J(C) = H^0(\omega_C)^\ast / H_1(C,\Z)$: choose a basis $\lambda_i$ ($i=1,\ldots,2g$) of $H_1(C,\Z)$, and $\omega_1, \ldots,\omega_g \in H^0(\omega_C)$. We have $\lambda_i = \sum_{j=1}^g (\int_{\lambda_i} \omega_j)\omega_j^\ast$, which span a $\Z$-lattice.

	The algebraic definition relates as follows:

	$$
	\Pic^0(C) \ni D \mapsto \{ \omega \mapsto \sum_{i=1}^N \int_{P_i}^{Q_i} \omega \} \pmod{H_1(C,\Z)}
	$$
	where we have written $D$ as a sum $\sum (P_i-Q_i)$.

	The \textbf{Abel-Jacobi theorem} states that this map is is well-defined and an isomorphism.


\subsubsection{Polarization} % (fold)
 \label{ssub:polarization}

 A \emph{polarization} on $X$ is an isogeny $\lambda: X \to X^\vee$ such that $\lambda=\lambda_{\mathcal L}$ for some ample sheaf $\mathcal L$.

 Here $\lambda_{\mathcal L}$ is the map $X \to X^\vee$ given by $a \mapsto (t_a^\ast \mathcal \otimes \mathcal L^{-1})$.
 
 % subsubsection polarization (end) 

\subsection{Theta characteristic on a curve $C$}

	A line bundle $\mathcal L$ such that $\mathcal L^{\otimes 2} \approx \omega_C$. It is called even/odd if $h^0(\mathcal L)$ is even/odd.

\subsection{Prym variety}

	First definition: let $\pi: \widetilde C \to C$ be an étale double covering of smooth algebraic curves. Let $\iota: \widetilde C \to \widetilde C$ be the involution switching the two sheets of $\pi$. Then $\sigma$ acts on $J(\widetilde C)$. Then $P \stackrel \Delta = \mathrm{im}(1-\sigma^\ast) \subset J(\widetilde C)$, the ``odd'' part of $J(\widetilde C)$. This is also equal to the connected component of the identity of the kernel of the ``norm map'' $\mathrm {Nm}:J(\widetilde C) \to J(C)$.




\end{document}
